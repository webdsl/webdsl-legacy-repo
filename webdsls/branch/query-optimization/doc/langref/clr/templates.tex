\chapter{Page and template definition}
WebDSL applications can be viewed as a set of defitions. In this chapter we will look into the \emph{page}\index{page} and \emph{template}\index{template} definitions. Page and templates definitions define directly the contents shown to the end-user and the behaviour he will expierence. Those definitions define the looks and the flow of your application. 

Pages and templates have a very similar structure. The semantics however are slightly different. First of all, \emph{pages} are directly refereable to with a URL. Second, a user is always at one page at a time. Navigation or program flow is achieved by moving from one page to another\footnote{However enabling Ajax in your applications might change this slightly}. A page can include several templates, as can templates themselves. However pages cannot be loaded inside other pages. 

\emph{Templates} on the other hand merely function as container to gather a set of \emph{elements} (see next chapter). Templates can call each other to enable reuse as much as possible in WebDSL applications. 

Both pages and templates can take a list of \emph{arguments} that the caller of a page/ template needs to provide. Overloading parameters is allowed. The syntax of pages and templates is as follows
\begin{webdsl}{Definition syntax}
defintion 	:= "define" modifier* name "(" formalargs ")" "{" elements* "}"
formalargs 	:= ( formalarg "," )*
formalarg		:= name ":" type
\end{webdsl}
Types and elements will be explained in the next chapters. Names are identifiers as we known them in programming languages like C en Java. Modifiers influence the type of object we are defining. This chapter introduced the modifiers \emph{page} and \emph{template}. If the modifier is omitted, \emph{template} is assumed. Other valid modifiers are \emph{email}, \emph{feed} and \emph{local}. They will not be explained here however. A simple page definitions might look like this:
\begin{webdsl}{Simple page definition}
define page user(u : User) {
	"The name of this user is " u.name
}
\end{webdsl}

\section{Redefinitions}\index{redefinitions}
Furthermore it is allowed for a page or template to redefine a template call. This allows to reuse a template while redefining a small part of it. This mechanic is shown in the example below. Both the \emph{home} and the \emph{publication} page call the \emph{main} template. This main template introduces a menu, a sidebar and a (not necessarily) empty body. This way both \emph{home} and \emph{publication} get those elements for free with the body at the proper place. However they still are able to define their own body. 

\begin{webdsl}{Redefinition example}
application redefine
section home page

define page home() {
  main()
  define body() {
  	//some other elements
  }
}

define page publication(p : Publication) {
  main()
  define body() {
  	//some completely other elements
  }
}

section templates 

define main() 
{ 
  sidebar() body() manageMenu() 
}  

define body() { //empty }
define manageMenu() { "menu" }
define sidebar() {"sidebar"}
\end{webdsl}
