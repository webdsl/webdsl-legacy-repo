% Getting Started
% 
% Installation
% 
% See the installation page.
% Using WebDSL for the first time
% 
% The webdsl generator script can generate a default "hello world" application:
% 
% $ webdsl new
% 
% This can be built using the build command:
% 
% $ webdsl build
% 
% This can then be deployed using the deploy command (replacing any existing deployment):
% 
% $ webdsl deploy
% 
% To clean all generated files:
% 
% $ webdsl clean
% 
% For further documentation of the webdsl script command, see webdsl help.
% 
% To run JBoss, enter the JBoss /bin/ directory and enter the following command:
% 
% $ export JAVA_OPTS="-server -Xms40m -Xmx1024m -XX:MaxPermSize=256m -XX:+CMSPermGenSweepingEnabled
%                     -XX:+CMSClassUnloadingEnabled -Xverify:none -Xss10m"
% $ ./run.sh -b 127.0.0.1 -Dbind.address=127.0.0.1
% 
% This runs JBoss in a Java Virtual Machine configured to handle the memory usage involved with (re)deploying large applications. Still, it is possible to receive out-of-memory errors in JBoss; memory leaks unfortunately seem to be a common problem with the application server.
% Documentation
% 
% See the language documentation for more information on the WebDSL language.
% Support
% 
% Found any bugs? Questions? Try the issue tracker, or see the home page for contact information. The developers usually hang out in the #webdsl irc channel on irc.freenode.net (web version), they will gladly help you out in getting started with webDSL and are able to give you up-to-date info on the status of the whole project (i.e. which revision you should use ;) ).

