% Types in WebDSL
% 
% String
% Represents a string of characters. Example:
% 
%     var s : String := "Hello world";
% 
% Int
% Represents an integer number. Example:
% 
%     var i : Int := 3;
% 
% Float
% Represents a floating number. Example:
% 
%     var f : Float := 3.5;
% 
% Bool
% Represents a truth. Either true or false. Example:
% 
%     var b : Bool := true;
% 
% Secret Represents a secret string (usually a password). Has two methods:
% 
%     * secretVar.check(digest) checks the secretVar password against the digest version contained in digest.
%     * secretVar.digest() generates a digest version of the clear-text password contained in secretVar.
% 
% Example:
% 
%     if (user.password.check(password)) {
%       securityContext.principal := us;
%       securityContext.loggedIn := true;
%     }
% 
% Email
% Represents an e-mail address as a string.
% 
% Text
% A longer string, in the UI represented as a text area and as outputed as formatted output.
% 
% WikiText
% 
% Wiki text represented as an extension of Text. The format of wiki text is based on MarkDown, with two extensions:
% 
%     * Wiki links: [ [page(MainPage)]] where "page" is the name of the page to link to and "MainPage" is the id of the first parameter. If no parameters are passed, simply [ [page] ] is also allowed. A wiki link can be given a different name like so: [ [page(Mainpage)|the main page]].
%     * Verbatim tag: < verbatim>Verbatim code< /verbatim>.
% 
% Date
% 
% Represents a date (not including a time). When input is used with a date type, a calendar widget will appear. Dates can be used as literals using the Date constructor:
% 
%     var d : Date := Date("22/06/1983");
% 
% Time
% 
% Represents a time (not including a date). Time literals can be expressed using a Time constructor. Example:
% 
%     var t : Time := Time("22:08");
% 
% DateTime
% 
% Represents both a date and a time. DateTime literals can be expressed using a DateTime constructor. Example:
% 
%     var dt : DateTime := DateTime("22/06/1983 22:08", "dd/MM/yyyy H:mm");
% 
% The second parameter is optional and can also be used for Date and Time types, it represents the date/time formatting string.
% 
% Patch
% 
% Represents a patch. Has two methods:
% 
%     * patch.applyPatch(objectString) applies patch to objectString
%     * str.makePatch(originalString) creates a patch from a variable of String-compatible type str, by comparing it to originalString.
% 
% Image
% 
% Represents an (uploaded) image.
% 
% File
% 
% Represents an (uploaded) file.
% 
% URL
% 
% Represents a link.
% 
% Set
% 
% Represents a collection of items of a certain type. Example
% 
%     var s : Set<Int> := {1, 2, 3, 4};
% 
% Fields:
% 
%     * s.length gives the number of items in this set.
% 
% List
% Represents an ordered list of items of a certain type. Example:
% 
%     var l : List<Int> := [1, 2, 3, 4];
% 
% Fields:
% 
%     * l.length gives the number of items in this list.
% 

