\chapter{WebDSL applications and their organization}

\section{The structure of a WebDSL application}
WebDSL application are organized in \emph{*.app}\index{.app files}\index{app files} files. Each .app file has a header, that either declares the name of a \emph{module}\index{module} or the name an \emph{application}\index{application}. The declared name should be identical to the filename. Each \emph{application} needs an .app file that declares the name of the application. This is the name refered to in the application.ini file (see below). 

An application can be organized in different \emph{modules}. In a typical .app file the header is followed by a serie of import statements, which contain a path to other modules (without extension). In this way your application can be seperated over several files, and modules can be reused. 

Within .app files one can define sections. A section is merely a label to indentify the structure of a file. Most sectionnames have no influence on the program itself, some have however, for example in styling definitions. 

The real contents of a .app file are a serie of definitions\index{definitions}. This might be page-, template-, action- or entity definitions. Other kinds of definitions might be introduced by WebDSL modules\footnote{A module might either refer to a module of an WebDSL application, or to an module of the WebDSL compiler itself. In this case the latter one is reffered to.}. Those definitions will be examined in detail in the next chapters. 

A very simple application might look like:
\begin{webdsl}{Helloworld.app}
application HelloWorld
imports MyFirstImport
section pages
define page home () { 
	"hello world" IAmImported() 
}
\end{webdsl}

\begin{webdsl}{MyFirstImport.app}
module MyFirstImport
section templates
define IAmImported() { 
	spacer "I am imported from a module file" 
}
\end{webdsl}

\section{Application configuration} 
\index{application configuration}\index{configuration}
In the \emph{application.ini} file compile-, database- and deployment information is stored. Executing the \emph{webdsl} command in a certain directory will look for a \emph{application.ini} file to obtain compilation information. If no such file was found, it will start a simple wizard to create one. 

The application.ini file contains the following information
\begin{description}
	\item[Appname]The name of the application compile. The compiler will look for a .app file named likewise and start compiling it
	\index{backend}
	\item[Backend]The target type of application. This can be either \emph{servlet} or \emph{python}. Older versions of WebDSL might also support \emph{seam} which generates JSF/Seam/JBoss servlets. 
	\item[Tomcatpath or JBosspath] Depending on the choice for backend, this field should contain the directory the compiled applications needs to be deployed or the proper backend is stored. For example \emph{/usr/bin/apache-tomcat}.
	\index{database configuration}
	\item[DBMode]This field indicates if the compiler should try to create a new database, or try to sync it with the new application to avoid loss of data. Valid values are \emph{create-drop} and \emph{update}. The latter one might lead to unpredictable results however if data model is changed to much. 
\end{description}
Furthermore there is a list of items needed to configure the database and email functionality. Those items speak for themselves. They are shown in the following example configurationfile:
\\application.ini:
\begin{shell}
export BACKEND=servlet
export TOMCATPATH=/opt/tomcat
export JBOSSPATH=
export APPNAME=HelloWorld
export DBSERVER=localhost
export DBUSER=michel
export DBPASSWORD=youdidliketoknowdontyou
export DBNAME=webdsldb
export DBMODE=create-drop
export SMTPHOST=localhost
export SMTPPORT=25
export SMTPUSER=
export SMTPPASS=
export SMTPSSL=false
export SMTPTLS=false
\end{shell}

