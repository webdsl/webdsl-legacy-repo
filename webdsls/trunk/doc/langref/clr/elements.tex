\chapter{User Interface elements}\index{user interface elements}
The contents of pages and templates are defined by templates elements. Roughly different kind of template elements exist. Control-flow elements (such as loops and variable declarations), User interface elements, actions and redefinitons. Redefinitions are mentioned in the previous chapter, control-flow elements and actions will be introduced in a later chapter. In this chapter we will focus on the variable declarations and user interface elements.

\index{HTML}User interface are some of the most concrete elements in WebDSL, as they closely resemble to the nature of HTML. Altough most elements can directly be mapped to HTML, some elements have extra semantics, such as headers. Furthermore the elements abstract from HTML, so in future it might be possible that other UI standards might be targeted (such as XUL, Gtk or Windows forms). 

\section{General structure}\index{attributes}
In general an interface elements consists of four parts. An element indication, a set of parameters, a set of attributes and a set of children. All currently available elements in WebDSL will be investigated in the next section. Attributes are static configuration properties, like an id declaration (so the element is uniquely refererable trough the whole application), a class declaration (to override style properties) and such things. As those attributes are particulary useful in an dynamic application, they will be explained in the \emph{modules - ajax} \ref{ajaxattributes} chapter. However some very useful properties will be mentioned already in this chapter. In general, attributes are not limited to a single kind of element, but are valid for different kind of objects.

Parameters are actual WebDSL objects which stronly influence the behaviour of an element. For example \emph{input} takes a data object to generate an input for, and \emph{navigate} takes a pagecall to generate a link. 

Some elements accept a group of children, which will be embedded in the object. However, not all elements take a group of children. Note that attributes and arguments are seperated from each other using comma's, but elements are seperated by just using whitespace. 
The general syntax of a interface element is:
\begin{webdsl}{WebDSL element syntax}
element	:= elementtype arguments? atributes? children?
arguments	:= "(" argument ")"
argument	:= expression ( "," argument )?
attributes	:= "[" attribute "]"
attribute := attributename ":=" attributevalue ( "," attribute )?
children	:= "{" element* "}"
\end{webdsl}
Expressions will be explained in a later chapter, valid attributenames and values will be explained whenever applicatble. In short this is what a set of UI elements look like:
\begin{webdsl}{Attribute example}
table[class:=maxwidth] {
  row {
    column {
      group("News") {
				//some newsitems
			}
		}
	}
}
\end{webdsl}

\section{Interface elements}
\subsection{Text}\index{text}
To output text, quoting it between '"' is enough. Escaping is done using '\' as is common in most languages. Note that due to the nature of HTML '\ n' has no meaning, except when used inside a \emph{pre} elements, which outputs text literally. A simple example:
\begin{webdsl}{Simple Text}
define simpleText {
	"This is some text.." "... and here is even more \"quoted\" text"
}
\end{webdsl}
Plain text will always be put inside a HTML \emph{span} element. To avoid this use \emph{text("some text")}.
%%TODO: heel onlogisch natuurlijk, dat zou eigenlijk omgekeerd moeten zijn: span e.d. zijn stijlbaar dus text attributen hebben dan zin. 

%%Elements are take from java servlet

\subsection{Basic elements}
\begin{description}
%title{ } -> page title
	\item[title{}]\index{title} Takes no arguments, but its contents is used to set the current page title. 
%header{ } with nesting and th's
	\item[header{}]\index{header} Generates a text header. The children of header are used as caption. The header depth is determined automatically, and so is the style. So a header element will result in a HTML \emph{h1} element, but if used inside a \emph{section} that already contains a header, it wll result in a HTML \emph{h2} element etc. 
%spacer, horizontal spacer -> horizontal line
	\item[spacer or horizontalspacer]\index{spacer}\index{horizontalspacer} results in a horizontal line (HTML: \emph{hr}). 
%image(string)
	\item[image(string)]\index{image} displays an image. The first argument is used as URL to the image, or can be a relative path from the applications directory to the image. Useful attributes for an image are \emph{height} and \emph{width}. Example: \emph{image("/images/logo.png")[height:= 100px, width:=20\%]}
%pre{} -> pre -> preformatted text
	\item[pre{}]\index{pre}\index{pre-formatted text} is used to display preformatted text. The text inside will be literally shown by the browser, using an alternative font and maintaining whitespace. Especially useful to display sourcecode. 
%label(string)
%label(string){}
	\item[label(string){}]\index{label} is used to render a HTML \emph{label}. The caption is determined by the first argument. If the label has any children, the label will link to the first child (clicking the label will give the focus to that child). If used inside a \emph{row} or \emph{group} element, the label and its children will be rendered in different columns. This way it is very easy to align different input elements and their captions. Example:
\begin{webdsl}{Group example}
group("edit "+user.name) {
	groupitem{label("name") 		{ input(user.name)} }
	groupitem{label("password) 	{ input(user.password) } }
}
\end{webdsl}
\end{description}


\subsection{Tables}
\begin{description}
%table{}
	\item[table{}]\index{table} defines a tabular structure. Tables are useful to display a set of simalar data (records), or to achieve a column-based layout. Table accepts any kind of element as child but every child that isn't a row will automatically wrapped by one. In other words putting elements directly in a table will order them below each other. It is advisable to explicitly use \emph{row} elements to avoid this implicit behaviour. A common usage pattern is to open a \emph{table} element, open a \emph{for} loop (see next chapter) to iterate over some data and display that data inside a \emph{row} element. See the example below. 
%row{} or nothing
	\item[row{}]\index{row} starts a new row. Is only useful in the context of a table, a control-flow element or as root element of a template (which can be called inside a table). Every element inside a row will be wrapped automatically with a column element if it isn't already one. 
	\item[column{}]\index{column}\index{cell} indicates a new column or \emph{cell}. Every row inside a table should have an equal number of columns to have no unpredicatble output. Use the rowspan or colspan attributes merge multiple cells. 
%column{} or column{header{}} or nothing
	\item[header{}]\index{header, inside table} results in a HTML \emph{th} element, indicating an alternative layout. Using \emph{row{header{...}}} or \emph{row{column{header{..}}}} both results in a single cell acting as header. 
\end{description}
The following code fragment displays a list of users. Note that all column elements could be omitted to achieve the same effect. However to force multiple elements to be displayed in one cell, the column element is necessary. 
\begin{webdsl}{Table example}
table {
	row { column { header { "name" } } column { header { "nickname" } } }
	for(u: User) {
		row { 
			column { output(u.name) } 
			column { output(u.nickname) }
		}
	}
}
\end{webdsl}

\subsection{Grouping elements}
Besides the table structure, WebDSL has a number of other elements to group elements. 
\begin{description}
%block{}, block(string){}, block(string,string){} -> div
	\item[block(string? string?){}]\index{block} is used to group a set of elements. Blocks are typically used to position elements
	using classes (see \emph{modules - styling and layout}). Beside that, blocks are useful to replace or update pages partially (see \emph{modules - ajax}). For that reason in practice the \emph{id} attribute is set on many blocks. A block takes zero, one or two arguments. All arguments should be strings and are used to assign styles to a block. All blocks result in HTML \emph{div} elements. 
	%section{} -> span with level semantics
	\item[section{}]\index{section} result in a HTML \emph{span} elements, just as plain text does. However section can be used to group multiple items inside a single span. This might be useful to assign a class attribute to multiple items. In contrast to blocks, sections are inlined. Furthermore sections keep track of the nesting depth, this is very useful in combination with the \emph{header} element. 
	%container{} -> span, == section? TODO: remove?
	\item[container{}]\index{container} result in a HTML \emph{span} element too, but without the nesting semantics of sections. 
	%par{}
	\item[par{}]\index{par} is used to create text paragraphs, and result in a HTML \emph{p} element. 
%list{} -> ul
%listitem{} -> il
	\item[list{} and listitem{}]\index{list}\index{listitem} are used to display lists. By default each item receive a bullet and all items are displayed below each other. This is completely configurable however using styling. To achieve predictable results, it is recommended to use only listitems as ui-children in list, and use only text based ui-elements in listitems (e.g. text, navigate, actionLink etc). 

%group(captionstring?){}
%groupitem(captionstring)
	\item[group(string){} and groupitem{}]\index{group}\index{groupitem} are very useful to group a set of elements visually, so the end-user notice a natural grouping of a set elements concerning the same issue. By default a group element renders a border and the first argument is used as caption. However it is allowed to provide no caption. The children of a group should be groupitems, otherwise each element will be wrapped by one. If a groupitem contains multiple children, each child results in a column, similar to the table structure. 
\end{description}

\subsection{Template calls}
Templates (see previous chapter) can be called as being an ordinary element. We saw this already in the chapter about pages and templates. Templatecalls take arguments (the parenthesis are mandatory) but take no children or attributes. Using templates we can rewrite the table listing earlier in this chapter to the following:
\begin{webdsl}{Template call example}
define template showUser(u: User) { 
	column { output(u.name) } 
	column { output(u.nickname) }
}

define page home() {
	table {
		row { column { header { "name" } } column { header { "nickname" } } }
		for(u: User) {
			row { showUser(u) }
		}
	}
}
\end{webdsl}

\subsection{Navigation elements}\index{navigation}\index{navigation, static}\index{link}
WebDSL knows two types of elements to define how an user is able to navigate trough an static application. Navigates an actions. Navigates replaces the current page with another page onces selected, action links to an action, which should return page that will be displayed after execution of the action (see next section). For dynamic applications more ways exist to navigate trough an application. See \emph{modules - Ajax}. 
\begin{description}
%navigate{} not refering link
%navigate(pagecall) {}referinglink
%navigate(urlstring){}
	\item[navigate(target){}]\index{navigate} creates a link (or \emph{anchor (a)} in HTML terms). A navigate takes zero or one argument. The argument can either be a string containing an URL or a pagecall (for example \emph{navigate(home()){"click me!"}}. Clicking on one of the children of a navigate will result in navigating to the provided location. The navigate without argument is only useful in combination with Ajax. 
%navigatebuttion(pagecall) //no children!
	\item[navigatebutton(target, string)]\index{navigatebutton} has the same semantics as a navigate, but takes two arguments and displays a button instead of a random element. The first argument again is the target location and the second argument is a string used as caption for the button. 
\end{description}

\subsection{Forms}\index{forms}\index{input}\index{user input}
\begin{description}
	\item[form{}]\index{form} is used to group a set of user inputs that should be submitted simultaniously to the server. Submitting usually done by creating a button that executes an action containing an submit. An example of a form is shown below. Inputs are created using the input element, explained in the next section. 
%action(string)
%action(string, actioncall)
	\item[action(string, actioncall?)]\index{action} shows a button which executes an action defined on the same page or template. The first argument is the caption of the button, the optional second argument is the call to the action to be executed. 
%actionLink(string)
%actionLink(string, actioncall)
	\item[actionLink(string, actioncall?)]\index{actionLink} takes the same arguments as \emph{action} and has the same behaviour, but shows a link rather than a button. 
\end{description}

A simple form might look like the following listing. Note that there is a distinction between the action element (\emph{action("remove", remove())}) and the action definition (\emph{action remove() { \ldots }}). Action definitions will be explained in a later chapter. 
\begin{webdsl}{Simple form example}
define editFolder(f: Folder) {
  form {
    group("editing folder "+f.name) {
      groupitem{
      	label("name ") {input(f.name) }
      }
      groupitem{
      	label("description") {input(f.description) }
      }
      groupitem {
      	action("remove", remove())
        action("save", save())
      }
    }
  }
  action save() {
    f.save();
    return showfoldercontents(f);
  }
  action remove() {
    f.delete();
    return showfolderlist();
  }
}
\end{webdsl}

\subsection{Generitave elements input and output}\index{user-input}\index{user-output}
There are two basic constructions to input and output data: \emph{input(object)} and \emph{output(object)}. Output is comparable to Java's \emph{toString()}\index{toString} method. The type of output is determined compile-time on the type of the variable provided. For example, a String variable results in a plain string being outputted, but a Image object results in a \emph{image} element being created. 
The same holds for the input element. Providing a String variable as input results in a simple textbox, providing an Image variable results in an uploadbox being created. 
It is possible to write your own input or output overload for a specific type by defining a template named \emph{input} or \emph{output} that accepts the proper type as first parameter. For example
\begin{webdsl}{Redifine output example}
define output(users: List<User>) {
	list {
		for(u: users) {
			listitem {
				output(u.nickname) ":" output(u.name)
			}
		}
	}
}
\end{webdsl}

\subsubsection{Specialized input and output elements}
Every \emph{input} and \emph{output} element will be rewritten to a more specialized input or output element. Those elements can be called directly instead of using \emph{input} or \emph{output}. This way the default inputfield for example can be replaced by a textarea. For list input possiblities, it is important to take a look at the \emph{control-flow} elements in the next chapter. This are all the input and output elements available by default:
\begin{description}
	\item[inputBool(boolean)]\index{inputBool}\index{checkbox} results in a checkbox being created. 
	\item[outputBool(boolean)]\index{outputBool} results in a disabled checkbox (with the proper state as one might expect)
	\item[inputString, outputString]\index{string}\index{inputString}\index{outputString} are used in combination with objects of type \emph{String}. The input displays a simple textfield. The output can also be used in combination with objects that are not of type String, but results might be unpredictable. 
	\item[inputInt, outputInt]\index{int}\index{inputInt}\index{outputInt} are used in combination with objects of type \emph{Int}. The input displays a simple inputbox. 
	\item[inputFloat, outputFloat]\index{float}\index{inputFloat}\index{outputFloat} \emph{are not implemented yet}
	\item[inputDate, inputTime, inputDateTime, outputDate, outputTime and outputDateTime]\index{Date}\index{Time}\index{DateTime}\index{inputDate}\index{inputTime}\index{inputDateTime}\index{outputTime}\index{outputDate}\index{outputDateTime} all take one argument, a expression of the primitive type Date, Time or DateTime and act as one might expect. \emph{Not implemented in the current version of WebDSL}
	\item[inputSecret, outputSecret]\index{secret}\index{password}\index{inputSecret}\index{outputSecret} display a password inputfield, which do not show the charachters to the users, but asterixes. outputSecret display a fixed number of asterixes. The provided object should be of type \emph{String} or \emph{Secret}
	\item[inputUrl]\index{URL}\index{inputURL} is used in combination with the \emph{Url} type, but works similar to inputString.
	\item[url]\index{url}\index{outputURL} displays an URL rather than creating a link. %TODO: should be outputURL? should show link?
	
	\item[inputText]\index{textarea}\index{inputText} creates a multiline textarea in which a String type can be inputted. 
	\item[outputText]\index{outputText}\index{Markdown}Outputs text, not literally but using a \emph{Markdown processor}\footnote{A markdown processor transforms plain text to rich XHTML text. Users can identify headings etc. intuitively. For more information see for example http://daringfireball.net/projects/markdown/} to process the text to be displayed. 
	\item[inputEmail, outputEmail]\index{email}\index{outputEmail}\index{InputEmail} \emph{are not implemented yet}
	\item[inputFile, outputFile]\index{file}\index{file upload}\index{file download}\index{upload file}\index{download file}\index{inputFile}\index{outputFile}create either an uploadbox or a download link, to send or retreive a file. Both require a single argument of primitive type \emph{File}. However, downloading can also be achieved by using actions too, see the primitive types chapter. This way access control rules can be applied. 
	\item[inputImage, outputImage]\index{image}\index{image upload}\index{image display} are used in combination with objects of the type \emph{Image}. The input creates an uploadbox, but the output displays the image. To download the image use a action that calls the \emph{download()} method on the image variable. 
	\item[inputWikiText, outputWikiText]\index{Wiki}\index{wikitext}\index{inputWikiText}\index{outputWikiText} are used to create Wiki like pages. The input again is a textarea, the output uses a Wiki text parser te generate output. WikiText elemens can be used on \emph{String} and \emph{WikiText} objects. 
\end{description}

\subsection{Menu's}\index{menu's}
\begin{description}
%menubar("horizontal" | "vertical" | nothing) {  }
	\item[menubar(direction?){}]\index{menubar} is used to start a menu. The parameter should contain \emph{"horizontal"}, \emph{"vertical"} or be omitted. Horizontal is the default. As children it is adviced to use \emph{menu} items, others are allowed but might break styling. 
%menu{ }
	\item[menu{}]\index{menu} is used to define one menu in a menubar. It should contain at least one element (menuheader) and optionally more menuitems in which case it will unfold if selected. 
%menuheader{//navigates}
	\item[menuheader{}]\index{menuheader} indicates the first element in a menu. It is advised to use only text or \emph{navigate} elements inside it. 
%menuitem{}
	\item[menuitem{}]\index{menuitem} represents a single item inside a menu.  It is advised to use only text or \emph{navigate} elements inside it. 
	\item[menuspacer] draws a line inside the current menu, to enable some grouping inside menu's. 
%menuspacer
\end{description}

