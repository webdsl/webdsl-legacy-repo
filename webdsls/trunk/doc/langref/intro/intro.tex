\section[]{An introduction to Domain Specific Languages}
%%Much taken from            WebDSL: A Case Study inDomain-Specific Language Engineering

Software Engineering can be summarized as the struggle for automating or optimizing the software development process. Main goals of software engineering are the improvement of development time, reducing the efforts in software maintanaince and increasing software quality. Since the early days of computer science abstractions have been build on top of other abstractions, leading away from the details of processor instructions, memory management etc. etc., in an attempt to reduce the number of concerns a software developer has to deal with. Those abstraction have led to the huge amount of GPL languages nowadays exist in industry, like Pascal, C, Python, Java and many many more. 
\\
To improve the improve the capabilities all GPL languages support notions like 'libaries' and 'API's', to enable reuse of specific functionality. However a GPL languages have a set of restrictions, which restrain to easily abstract from the GPL, leaving the developer with lots of boilerplate code that has to be written (like initializing database transactions, write conversion mechanims etc). The restrictions created by the nature of GPL's can be summarized as follow:
\\
\begin{description}
	\item[The inexstensibilty of syntax]
	Libraries offer the capability to automate specific semantic behaviour, like querying a database. However, no syntactic extensions or embeddings are possible, to be able to define SQL queries in an intuitive manner. One cannot avoid the usage of strings or other kinds of code-sugar. 
	\item[The absence of domain restrictions]
	In a GPL one can express in every function or class everything that can be expressed in the GPL as a whole. This makes GPL's very expressive in general, it always enables unstructured, semantic incorrect or very indepent code.
	\item[The huge amount of code needed for general patterns]
	Within a certain domain, certain actions or patterns have to be executed frequently. Little abstractions can be made using helper or library functions, inheritance or even macro's to reuse as much code as possible. However the ways abstractions can be made are limited, and not every abstraction can be expressed in the language
	\item[The huge depencens on the target GPL]
	Once a application has been developed in a certain GPL, it takes much effort to rebuild the application in another GPL, even when the language abstractions, semantics and syntax are very similir (e.g. Java and C\#). This show how much code relies on the language it is written for, even when the abstractions and semantics required by that language are very similar to the ones made available with another GPL. 
\end{description}

To deal with the probles stated above, much effort has been put in Model Driven Engineering. Goal is to have a model describing an application, abstraction from the details of programming languages. Domain Specific Languages provide an interface to a developer where only behaviour related to the domain should be specificied, while the compiler, code generator or model interpreter takes care of the boilerplate code needed for a fully working application, query or whatever the target of a DSL. In [dezelfde paper dus] a DSL is defined as
\begin{quote}
A domain-specific language (DSL) is a high-level software implementa
tion language that supports concepts and abstractions that are related
to a particular (application) domain.\footnote{TODO}
\end{quote}
\section[]{Purpose of WebDSL}
WebDSL is an attempt to create a domain specific language for web-applications. Things like peristency, request handling and GPL/ Markup code generation will be taken care of by the language compiler. The responsiblility of the user is to think about the data model, navigation, page layout and such issues. No boilerplate code should be needed to be written by the user, and the way to define te model (textual in this case) should feel intuitive. The purpose of WebDSL is stated at WebDSL.org as follows
\begin{quote}
WebDSL is a domain-specific language for developing dynamic web applications with a rich data model.
\footnote{TODO}
\end{quote}

The language supports modeling of
\begin{itemize}
	\item the applications domain in data entities
	\item the applications user interface
	\item the page flow
	\item security and access restrictions
	\item dynamic behaviour, commonly refered to as Ajax. 
	\item styling
\end{itemize}

Last but not least, webDSL applications can be targeted to different front- and backends, including a Tomcat servlet, Google Apps Python backend and a Ajax enabled (dynamic pages) or Ajax-disabled (static pages) front-end. 

\section[]{Scientific work}

lijst met verschenen papers
% WebDSL
% 
% Feedback
% 
% WebDSL is a domain-specific language for developing dynamic web applications with a rich data model.
% Features
% 
%     * Domain modeling
%     * Presentation
%     * Page-flow
%     * Access control
%     * Workflow (under construction)
% 
% Software
% 
%     * WebDSL applications are translated to Java webapplications, building on the JSF, Hibernate, and Seam frameworks.
%     * The WebDSL generator is implemented using Stratego/XT and SDF.
%     * Deployment is realized with the Nix software deployment system.
% 
% Release
% 
%     * First alpha release December 2007.
%     * Download
%     * Give us feedback
% 
% Course
% 
% This year's course on program transformation and generation treats WebDSL as a case study of a program generator.
% 
% More details can be found on the program transformation course page.
% Documentation
% 
%     * Getting started
%     * Installing WebDSL
%     * Language documentation
% 
% Publications
% 
%     *
% 
%       E. Visser. "Domain-Specific Language Engineering." In R. Lämmel and J. Saraiva, editors, Proceedings of the Summer School on Generative and Transformational Techniques in Software Engineering (GTTSE'07), Lecture Notes in Computer Science. Springer Verlag, Braga, Portugal, July 2007. (invited tutorial; under construction)
%     *
% 
%       Z. Hemel, L. C. L. Kats, and E. Visser. "Code Generation by Model Transformation" In International Conference on Model Transformation (ICMT'08).
%     *
% 
%       D. Groenewegen and E. Visser. "Declarative Access Control for WebDSL: Combining Language Integration and Separation of Concerns" In International Conference on Web Engineering (ICWE'08) Yorktown Heights, New York, July 2008.
% 
% Developers
% 
% WebDSL is being developed by Eelco Visser and (Ph.D.) students in the context of the Model-Driven Software Evolution project at Delft University of Technology.
% 
%     * Eelco Visser
%     * Zef Hemel (Zef's Development Blog)
%     * Danny Groenewegen
%     * Jippe Holwerda
%     * Lennart Kats
%     * Sander Vermolen
%     * Sander van der Burg
% 

