This section describes how to install WebDSL if it is only to be used to compile WebDSL applications. Currently the compiler is only available for Unix based systems, like Linux and Apple's OS-X. The required packages depend on the target platform of WebDSL applications. The WebDSL compiler itself has the following dependencies:
\begin{itemize}
	\item The \emph{gcc} C compiler
\end{itemize}

\section{A note on Nix deployment system}\label{installnix}
\index{Nix deployment system}
WebDSL and Stratego/XT (one of the depencies when extending WebDSL) can be installed manually or using the \emph{Nix}\footnote{http://www.nixos.org} deployment system, which is highly recommended. Nix takes care of depencies and provides the functionality to easily obtain the latest version from the buildfarms. The Nix deployment system does not interfere with other package managers on your system, like aptitude. 

The latest release of Nix can be found at its website \emph{http://www.nixos.org}. Use the \emph{configure}, \emph{make} and \emph{make install} commands to install the deployment system. To be able to fully use and manage the applications installed by Nix, make your user owner of the \e{/nix} directory:
\begin{shell}
sudo chown -R username:groupname /nix/
\end{shell}
As a final step a autostart instruction needs to be added to your \emph{\tilde/.profile} file (a re-login might be required before the changes are applied):
\begin{shell}
. /nix/etc/profile.d/nix.sh
\end{shell}

\section{Installing the WebDSL compiler}\label{retreivewebdsl}
Two ways exist to obtain WebDSL, either from the Nix channel, or downloading the sources directly. The latest source can be downloaded from \\\emph{http://buildfarm.st.ewi.tudelft.nl/releases/strategoxt/full-index-webdsls.html\#webdsls-Unstable}.

\index{WebDSL deployment channel}\label{updatewebdsl}
Using Nix, one can subscribe to the WebDSL channel using the following code. Replace \e{webdsl-version} with the name of the most recent version of the \e{webdsls} package at \e{http://buildfarm.st.ewi.tudelft.nl/releases/strategoxt/}
\begin{shell}
nix-channel --add http://buildfarm.st.ewi.tudelft.nl/releases/strategoxt/channels-v3/webdsls-unstable
nix-channel --update
nix-env -i webdsl-version
\end{shell}
After subscribing to the channel updating to the latest version can be done simply using the following commands:
\begin{shell}
nix-channel --update
nix-env -u webdsls
\end{shell}
After obtaining the sources, executing \emph{./bootstrap} and the well-known commands \emph{./configure}, \emph{./make} and \emph{./make install} should do the job. Test your installation by building a first WebDSL application, as described in \ref{firstapp}. Note that your test application cannot be deployed until the instructions in the next section are executed. Make sure your system has common compile tools like \e{g++}, \e{gcc}, \e{make}, \e{ant}, \e{docbook} etc available. The \e{./bootstrap} command is not required when \e{Stratego/XT} was not updated. 

\subsection{Installing the WebDSL backends}\label{installbackend}
Currently, webdsl applications can be compiled to Java Servlets hosted by a Tomcat server\footnote{http://tomcat.apache.org} or to a Python script which can be hosted by Google's AppEngine\footnote{http://code.google.com/appengine/}. 

\subsubsection{Installing the Tomcat backend}
\index{backend, Tomcat}

\subsubsection{Installing the Python backend}
\index{backend, Python}
