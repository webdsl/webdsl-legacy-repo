%THanks to M. Bravenboer:
\documentclass[10pt, b5paper,twoside,openright]{book}
\usepackage[nohead]{geometry}
\geometry{vmarginratio=2:3,hmarginratio=1:1,papersize={170mm,240mm},total={121mm,196mm}}

\usepackage{loadpackages}
\usepackage{style-tweaks}
\usepackage{color}
\usepackage{xcolor}
\usepackage{listings}
\input{commands.ltx}
%\definecolor{darkgreen}{rgb}{0,0.4,0}
\definecolor{darkblue}{rgb}{0,0,0.6}
\definecolor{darkred}{rgb}{0.6,0,0}
\definecolor{lightgray}{rgb}{0.9,0.9,0.9}
\lstdefinestyle{mystyle}
  {
  basicstyle=\ttfamily,%
  showstringspaces=false,
  keywordstyle=[1]{\color{black}\bfseries},%
  keywordstyle=[2]{\color{darkblue}\bfseries},%
  keywordstyle=[3]{\color{blue}\emph},
  keywordstyle=[4]{\color{darkred}\emph},
 	keywordstyle=[5]{\color{green}},
  stringstyle={\color{black}},%
  commentstyle={\color{darkgray}\emph},%
  emphstyle={\warn},%
  tabsize=4,
  aboveskip=10pt,
  belowskip=10pt,
  tabsize=4,
  breaklines=true,
  breakautoindent=true,
  frame=none
}
%
\lstdefinelanguage{webdsl}
  {	basicstyle=\ttfamily,
  	style=mystyle,
  	morekeywords=[1]{template, page, entity, feed, action, function},	
  	morekeywords=[2]{init,define, using, section, application, module, rule, description, note},
		morekeywords=[3]{inverse,where,order,by,asc,desc},
		morekeywords=[4]{for,select,if, in,retur,cancel,goto,var,else,is,a,as},	
		morekeywords=[5]{true,false,null},
    sensitive=false,
    morecomment=[l]{//},
    morecomment=[s]{/*}{*/},
    morestring=[b]",
    morestring=[d]’,
    literate={|[}{{\texttt{|$\!\!$[}}}1%
	    {]|}{{\texttt{]$\!\!$|}}}1%
	    {->}{$\rightarrow$ }2%
	    {=>}{$\Rightarrow$ }2,
	  numbers=right,  	    
   moredelim=[is][basicstyle]{`}{`} % to escape non-keyword words
  }
\lstnewenvironment{webdsl}[1]{\lstset{language=webdsl,caption=#1, captionpos=top}}{}
%
\lstdefinelanguage{shell}
   {
   	style=mystyle,
     morecomment=[l]{//},
     morecomment=[s]{/*}{*/},
     morestring=[b]",
     morestring=[d]’
   }
\lstnewenvironment{shell}
     {\lstset{language=shell, frame=leftline,  xleftmargin=20pt}}
     {}

%\definecolor{darkgreen}{rgb}{0,0.4,0}
\definecolor{darkblue}{rgb}{0,0,0.6}
\definecolor{darkred}{rgb}{0.6,0,0}
\definecolor{lightgray}{rgb}{0.9,0.9,0.9}
\lstdefinestyle{mystyle}
  {
  basicstyle=\ttfamily,%
  showstringspaces=false,
  keywordstyle=[1]{\color{black}\bfseries},%
  keywordstyle=[2]{\color{darkblue}\bfseries},%
  keywordstyle=[3]{\color{blue}\emph},
  keywordstyle=[4]{\color{darkred}\emph},
 	keywordstyle=[5]{\color{green}},
  stringstyle={\color{black}},%
  commentstyle={\color{darkgray}\emph},%
  emphstyle={\warn},%
  tabsize=4,
  aboveskip=10pt,
  belowskip=10pt,
  tabsize=4,
  breaklines=true,
  breakautoindent=true,
  frame=none
}
%
\lstdefinelanguage{webdsl}
  {	basicstyle=\ttfamily,
  	style=mystyle,
  	morekeywords=[1]{template, page, entity, feed, action, function},	
  	morekeywords=[2]{init,define, using, section, application, module, rule, description, note},
		morekeywords=[3]{inverse,where,order,by,asc,desc},
		morekeywords=[4]{for,select,if, in,retur,cancel,goto,var,else,is,a,as},	
		morekeywords=[5]{true,false,null},
    sensitive=false,
    morecomment=[l]{//},
    morecomment=[s]{/*}{*/},
    morestring=[b]",
    morestring=[d]’,
    literate={|[}{{\texttt{|$\!\!$[}}}1%
	    {]|}{{\texttt{]$\!\!$|}}}1%
	    {->}{$\rightarrow$ }2%
	    {=>}{$\Rightarrow$ }2,
	  numbers=right,  	    
   moredelim=[is][basicstyle]{`}{`} % to escape non-keyword words
  }
\lstnewenvironment{webdsl}[1]{\lstset{language=webdsl,caption=#1, captionpos=top}}{}
%
\lstdefinelanguage{shell}
   {
   	style=mystyle,
     morecomment=[l]{//},
     morecomment=[s]{/*}{*/},
     morestring=[b]",
     morestring=[d]’
   }
\lstnewenvironment{shell}
     {\lstset{language=shell, frame=leftline,  xleftmargin=20pt}}
     {}

\definecolor{darkgreen}{rgb}{0,0.4,0}
\definecolor{darkblue}{rgb}{0,0,0.6}
\definecolor{darkred}{rgb}{0.6,0,0}
\definecolor{lightgray}{rgb}{0.9,0.9,0.9}
\lstdefinestyle{mystyle}
  {
  basicstyle=\ttfamily,%
  showstringspaces=false,
  keywordstyle=[1]{\color{black}\bfseries},%
  keywordstyle=[2]{\color{darkblue}\bfseries},%
  keywordstyle=[3]{\color{blue}\emph},
  keywordstyle=[4]{\color{darkred}\emph},
 	keywordstyle=[5]{\color{green}},
  stringstyle={\color{black}},%
  commentstyle={\color{darkgray}\emph},%
  emphstyle={\warn},%
  tabsize=4,
  aboveskip=10pt,
  belowskip=10pt,
  tabsize=4,
  breaklines=true,
  breakautoindent=true,
  frame=none
}
%
\lstdefinelanguage{webdsl}
  {	basicstyle=\ttfamily,
  	style=mystyle,
  	morekeywords=[1]{template, page, entity, feed, action, function},	
  	morekeywords=[2]{init,define, using, section, application, module, rule, description, note},
		morekeywords=[3]{inverse,where,order,by,asc,desc},
		morekeywords=[4]{for,select,if, in,retur,cancel,goto,var,else,is,a,as},	
		morekeywords=[5]{true,false,null},
    sensitive=false,
    morecomment=[l]{//},
    morecomment=[s]{/*}{*/},
    morestring=[b]",
    morestring=[d]’,
    literate={|[}{{\texttt{|$\!\!$[}}}1%
	    {]|}{{\texttt{]$\!\!$|}}}1%
	    {->}{$\rightarrow$ }2%
	    {=>}{$\Rightarrow$ }2,
	  numbers=right,  	    
   moredelim=[is][basicstyle]{`}{`} % to escape non-keyword words
  }
\lstnewenvironment{webdsl}[1]{\lstset{language=webdsl,caption=#1, captionpos=top}}{}
%
\lstdefinelanguage{shell}
   {
   	style=mystyle,
     morecomment=[l]{//},
     morecomment=[s]{/*}{*/},
     morestring=[b]",
     morestring=[d]’
   }
\lstnewenvironment{shell}
     {\lstset{language=shell, frame=leftline,  xleftmargin=20pt}}
     {}


%end copied

\usepackage{ifpdf}
\usepackage[utf8]{inputenc}
%\usepackage[UKenglish]{babel}
\usepackage{makeidx}
\makeindex
\frenchspacing

\newcommand{\e}[1]{\emph{#1}}
\newcommand{\ii}[1]{\makeindex{#1}}

\title{WebDSL language reference}
\author{The WebDSL Team}
\date{\today}
\ifpdf
\pdfinfo {
	/Author (The WebDSL Team)
	/Title (WebDSL language reference)
	/Subject (WebDSL)
	/Keywords ()
	/CreationDate (D:20081020135807)
}
\usepackage{graphicx}
\fi

\begin{document}
\frontmatter
\maketitle
\tableofcontents 

\mainmatter
\part{Introduction to WebDSL}
\chapter{Introduction to WebDSL}
\section[]{An introduction to Domain Specific Languages}
%%Much taken from            WebDSL: A Case Study inDomain-Specific Language Engineering
\index{Domain Specific Language (DSL)}
Software Engineering can be summarized as the struggle for automating or optimizing the software development process. Main goals of software engineering are the improvement of development time, reducing the efforts in software maintanaince and increasing software quality. Since the early days of computer science abstractions have been build on top of other abstractions, leading away from the details of processor instructions, memory management etc. etc., in an attempt to reduce the number of concerns a software developer has to deal with. Those abstraction have led to the huge amount of GPL languages nowadays exist in industry, like Pascal, C, Python, Java and many many more. 
\\
To improve the improve the capabilities all GPL languages support notions like 'libaries' and 'API's', to enable reuse of specific functionality. However a GPL languages have a set of restrictions, which restrain to easily abstract from the GPL, leaving the developer with lots of boilerplate code that has to be written (like initializing database transactions, write conversion mechanims etc). The restrictions created by the nature of GPL's can be summarized as follow:
\\
\begin{description}
	\item[The inexstensibilty of syntax]
	Libraries offer the capability to automate specific semantic behaviour, like querying a database. However, no syntactic extensions or embeddings are possible, to be able to define SQL queries in an intuitive manner. One cannot avoid the usage of strings or other kinds of code-sugar. 
	\item[The absence of domain restrictions]
	In a GPL one can express in every function or class everything that can be expressed in the GPL as a whole. This makes GPL's very expressive in general, it always enables unstructured, semantic incorrect or very indepent code.
	\item[The huge amount of code needed for general patterns]
	Within a certain domain, certain actions or patterns have to be executed frequently. Little abstractions can be made using helper or library functions, inheritance or even macro's to reuse as much code as possible. However the ways abstractions can be made are limited, and not every abstraction can be expressed in the language
	\item[The huge depencens on the target GPL]
	Once a application has been developed in a certain GPL, it takes much effort to rebuild the application in another GPL, even when the language abstractions, semantics and syntax are very similir (e.g. Java and C\#). This show how much code relies on the language it is written for, even when the abstractions and semantics required by that language are very similar to the ones made available with another GPL. 
\end{description}

To deal with the probles stated above, much effort has been put in Model Driven Engineering. Goal is to have a model describing an application, abstraction from the details of programming languages. Domain Specific Languages provide an interface to a developer where only behaviour related to the domain should be specificied, while the compiler, code generator or model interpreter takes care of the boilerplate code needed for a fully working application, query or whatever the target of a DSL. In [dezelfde paper dus] a DSL is defined as
\begin{quote}
A domain-specific language (DSL) is a high-level software implementa
tion language that supports concepts and abstractions that are related
to a particular (application) domain.\footnote{TODO}
\end{quote}
\section[]{Purpose of WebDSL}
WebDSL is an attempt to create a domain specific language for web-applications. Things like peristency, request handling and GPL/ Markup code generation will be taken care of by the language compiler. The responsiblility of the user is to think about the data model, navigation, page layout and such issues. No boilerplate code should be needed to be written by the user, and the way to define te model (textual in this case) should feel intuitive. The purpose of WebDSL is stated at WebDSL.org as follows
\begin{quote}
WebDSL is a domain-specific language for developing dynamic web applications with a rich data model.
\footnote{TODO}
\end{quote}

The language supports modeling of
\begin{itemize}
	\item the applications domain in data entities
	\item the applications user interface
	\item the page flow
	\item security and access restrictions
	\item dynamic behaviour, commonly refered to as Ajax. 
	\item styling
\end{itemize}

\index{backend} \index{target}
Last but not least, webDSL applications can be targeted to different front- and backends, including a Tomcat servlet, Google Apps Python backend and a Ajax enabled (dynamic pages) or Ajax-disabled (static pages) front-end. 

\section[]{Scientific work}

lijst met verschenen papers
% WebDSL
% 
% Feedback
% 
% WebDSL is a domain-specific language for developing dynamic web applications with a rich data model.
% Features
% 
%     * Domain modeling
%     * Presentation
%     * Page-flow
%     * Access control
%     * Workflow (under construction)
% 
% Software
% 
%     * WebDSL applications are translated to Java webapplications, building on the JSF, Hibernate, and Seam frameworks.
%     * The WebDSL generator is implemented using Stratego/XT and SDF.
%     * Deployment is realized with the Nix software deployment system.
% 
% Release
% 
%     * First alpha release December 2007.
%     * Download
%     * Give us feedback
% 
% Course
% 
% This year's course on program transformation and generation treats WebDSL as a case study of a program generator.
% 
% More details can be found on the program transformation course page.
% Documentation
% 
%     * Getting started
%     * Installing WebDSL
%     * Language documentation
% 
% Publications
% 
%     *
% 
%       E. Visser. "Domain-Specific Language Engineering." In R. Lämmel and J. Saraiva, editors, Proceedings of the Summer School on Generative and Transformational Techniques in Software Engineering (GTTSE'07), Lecture Notes in Computer Science. Springer Verlag, Braga, Portugal, July 2007. (invited tutorial; under construction)
%     *
% 
%       Z. Hemel, L. C. L. Kats, and E. Visser. "Code Generation by Model Transformation" In International Conference on Model Transformation (ICMT'08).
%     *
% 
%       D. Groenewegen and E. Visser. "Declarative Access Control for WebDSL: Combining Language Integration and Separation of Concerns" In International Conference on Web Engineering (ICWE'08) Yorktown Heights, New York, July 2008.
% 
% Developers
% 
% WebDSL is being developed by Eelco Visser and (Ph.D.) students in the context of the Model-Driven Software Evolution project at Delft University of Technology.
% 
%     * Eelco Visser
%     * Zef Hemel (Zef's Development Blog)
%     * Danny Groenewegen
%     * Jippe Holwerda
%     * Lennart Kats
%     * Sander Vermolen
%     * Sander van der Burg
% 



\chapter[installuser]{Installing WebDSL for WebDSL users}
This section describes how to install WebDSL if it is only to be used to compile WebDSL applications. Currently the compiler is only available for Unix based systems, like Linux and Apple's OS-X. The required packages depend on the target platform of WebDSL applications. The WebDSL compiler itself has the following dependencies:
\begin{itemize}
	\item The \emph{gcc} C compiler
\end{itemize}

\section{A note on Nix deployment system}\label{installnix}
\index{Nix deployment system}
WebDSL and Stratego/XT (one of the depencies when extending WebDSL) can be installed manually or using the \emph{Nix}\footnote{http://www.nixos.org} deployment system, which is highly recommended. Nix takes care of depencies and provides the functionality to easily obtain the latest version from the buildfarms.

The latest release of Nix can be found at its website \emph{http://www.nixos.org}. Use the \emph{configure}, \emph{make} and \emph{make install} commands to install the deployment system. As a final step a autostart instruction needs to be added to your \emph{~.profile} file:
\begin{shell}
. /nix/etc/profile.d/nix.sh
\end{shell}

\section{Installing the WebDSL compiler}\label{retreivewebdsl}
Two ways exist to obtain WebDSL, either from the Nix channel, or downloading the sources directly. The latest source can be downloaded from \emph{http://buildfarm.st.ewi.tudelft.nl/releases/strategoxt/full-index-webdsls.html\#webdsls-Unstable}.

\index{WebDSL deployment channel}
Using Nix, one can subscribe to the WebDSL channel using the following code: 
\begin{shell}
nix-channel --add http://buildfarm.st.ewi.tudelft.nl/releases/strategoxt/channels-v3/webdsls-unstable
nix-channel --update
nix-env -i webdsl-8.3pre1057
\end{shell}
After subscribing to the channel updating to the latest version can be done simply using the following commands:
\begin{shell}
nix-channel --update
nix-env -u webdsls
\end{shell}
After obtaining the sources, executing the well-known commands \emph{configure}, \emph{make} and \emph{make install} should do the job. Test your installation by building a first WebDSL application, as described in \ref{firstapp}. Note that your test application cannot be deployed until the instructions in the next section are executed. 

\subsection{Installing the WebDSL backends}\label{installbackend}
Currently, webdsl applications can be compiled to Java Servlets hosted by a Tomcat server\footnote{http://tomcat.apache.org} or to a Python script which can be hosted by Google's AppEngine\footnote{http://code.google.com/appengine/}. 

\subsubsection{Installing the Tomcat backend}
\index{backend, Tomcat}

\subsubsection{Installing the Python backend}
\index{backend, Python}


\chapter{Installing WebDSL for WebDSL developers}
\index{Stratego/XT}
To build your own version of WebDSL, it takes some extra dependencies that needs to be satifisfied before being able to build the compiler. WebDSL has ben beeld on the program transformation toolset \emph{Stratego/XT}\footnote{http://www.strategoxt.org}. First off all, make sure the depencies of Stratego/XT are satisfied. The following linux packages are needed:
\begin{itemize}
	\item install 
	\item curl 
	\item m4 
	\item autoconf 
	\item automake 
	\item pkgconfig 
	\item libtool 
	\item subversion
\end{itemize}

\index{Stratego/XT channel}
Stratego/XT can be installed using the Nix distributed system. Make sure Nix is installed as described in \ref{installnix}. To install Stratego/XT open a console and execute the following commands:
\begin{shell}
nix-channel --add http://releases.strategoxt.org/strategoxt-packages/channels/strategoxt-packages-stable
nix-channel --add http://releases.strategoxt.org/strategoxt/channels/strategoxt-unstable
nix-channel --update
\end{shell}
Then install the stratego packages:
\begin{shell}
nix-env -i aterm java-front sdf2-bundle stratego-shell strategoxt strategoxt-utils
\end{shell}
Finally, update your \emph{~/.profile} again and add:
\begin{shell}
export PKG_CONFIG_PATH=~/.nix-profile/lib/pkgconfig
\end{shell}

\index{WebDSL subversion repository}
The full sources of WebDSL can be retreived from the svn repository at \emph{ https://svn.strategoxt.org/repos/WebDSL/webdsls}. Use the following command to obtain the sources:
\begin{shell}
svn co  https://svn.strategoxt.org/repos/WebDSL/webdsls/trunk
\end{shell}
Finally, you can compile your own webdsl compiler using 
\begin{shell}
cd webdsls
./bootstrap
./configure --disable-shared --prefix=/usr/local
make
make install
\end{shell}
Install any required backends as described in \ref{instalbackend} and check your installation as described in \ref{firstapp}

\index{Macintosh}
\subsection{A note on Mac Users}
Macintosh users require to install some applications already availalble in most linux environments. First install Apple \emph{XCode} from your DVD. Secondly download and install \emph{DarwinPorts}. Follow the instructions and then install the required ports:
\begin{shell}
port install curl m4 autoconf automake pkgconfig libtool subversion
\end{shell}


\chapter[firstapp]{Running your first application }
% Getting Started
% 
% Installation
% 
% See the installation page.
% Using WebDSL for the first time
% 
% The webdsl generator script can generate a default "hello world" application:
% 
% $ webdsl new
% 
% This can be built using the build command:
% 
% $ webdsl build
% 
% This can then be deployed using the deploy command (replacing any existing deployment):
% 
% $ webdsl deploy
% 
% To clean all generated files:
% 
% $ webdsl clean
% 
% For further documentation of the webdsl script command, see webdsl help.
% 
% To run JBoss, enter the JBoss /bin/ directory and enter the following command:
% 
% $ export JAVA_OPTS="-server -Xms40m -Xmx1024m -XX:MaxPermSize=256m -XX:+CMSPermGenSweepingEnabled
%                     -XX:+CMSClassUnloadingEnabled -Xverify:none -Xss10m"
% $ ./run.sh -b 127.0.0.1 -Dbind.address=127.0.0.1
% 
% This runs JBoss in a Java Virtual Machine configured to handle the memory usage involved with (re)deploying large applications. Still, it is possible to receive out-of-memory errors in JBoss; memory leaks unfortunately seem to be a common problem with the application server.
% Documentation
% 
% See the language documentation for more information on the WebDSL language.
% Support
% 
% Found any bugs? Questions? Try the issue tracker, or see the home page for contact information. The developers usually hang out in the #webdsl irc channel on irc.freenode.net (web version), they will gladly help you out in getting started with webDSL and are able to give you up-to-date info on the status of the whole project (i.e. which revision you should use ;) ).



%%Welcome, when you are reading this, you are at the point of solving a recursive problem!
\chapter{Compiling this documentation}
The sources of this documentation can be found in the ./doc/langref/ directory of the WebDSL sources. Compiling the document requires Latex[TODO] and Rubber[TODO] to be installed. After updating the WebDSL source to the most recent version, you may compile your own documentation postscript with the command
\lstset{language={},tabsize=4}
\begin{lstlisting}
rubber -f -s --inplace -d index.tex
\end{lstlisting}


\part{Core Language Reference}
\chapter{Introduction to WebDSL}
\section[]{An introduction to Domain Specific Languages}
%%Much taken from            WebDSL: A Case Study inDomain-Specific Language Engineering
\index{Domain Specific Language (DSL)}
Software Engineering can be summarized as the struggle for automating or optimizing the software development process. Main goals of software engineering are the improvement of development time, reducing the efforts in software maintanaince and increasing software quality. Since the early days of computer science abstractions have been build on top of other abstractions, leading away from the details of processor instructions, memory management etc. etc., in an attempt to reduce the number of concerns a software developer has to deal with. Those abstraction have led to the huge amount of GPL languages nowadays exist in industry, like Pascal, C, Python, Java and many many more. 
\\
To improve the improve the capabilities all GPL languages support notions like 'libaries' and 'API's', to enable reuse of specific functionality. However a GPL languages have a set of restrictions, which restrain to easily abstract from the GPL, leaving the developer with lots of boilerplate code that has to be written (like initializing database transactions, write conversion mechanims etc). The restrictions created by the nature of GPL's can be summarized as follow:
\\
\begin{description}
	\item[The inexstensibilty of syntax]
	Libraries offer the capability to automate specific semantic behaviour, like querying a database. However, no syntactic extensions or embeddings are possible, to be able to define SQL queries in an intuitive manner. One cannot avoid the usage of strings or other kinds of code-sugar. 
	\item[The absence of domain restrictions]
	In a GPL one can express in every function or class everything that can be expressed in the GPL as a whole. This makes GPL's very expressive in general, it always enables unstructured, semantic incorrect or very indepent code.
	\item[The huge amount of code needed for general patterns]
	Within a certain domain, certain actions or patterns have to be executed frequently. Little abstractions can be made using helper or library functions, inheritance or even macro's to reuse as much code as possible. However the ways abstractions can be made are limited, and not every abstraction can be expressed in the language
	\item[The huge depencens on the target GPL]
	Once a application has been developed in a certain GPL, it takes much effort to rebuild the application in another GPL, even when the language abstractions, semantics and syntax are very similir (e.g. Java and C\#). This show how much code relies on the language it is written for, even when the abstractions and semantics required by that language are very similar to the ones made available with another GPL. 
\end{description}

To deal with the probles stated above, much effort has been put in Model Driven Engineering. Goal is to have a model describing an application, abstraction from the details of programming languages. Domain Specific Languages provide an interface to a developer where only behaviour related to the domain should be specificied, while the compiler, code generator or model interpreter takes care of the boilerplate code needed for a fully working application, query or whatever the target of a DSL. In [dezelfde paper dus] a DSL is defined as
\begin{quote}
A domain-specific language (DSL) is a high-level software implementa
tion language that supports concepts and abstractions that are related
to a particular (application) domain.\footnote{TODO}
\end{quote}
\section[]{Purpose of WebDSL}
WebDSL is an attempt to create a domain specific language for web-applications. Things like peristency, request handling and GPL/ Markup code generation will be taken care of by the language compiler. The responsiblility of the user is to think about the data model, navigation, page layout and such issues. No boilerplate code should be needed to be written by the user, and the way to define te model (textual in this case) should feel intuitive. The purpose of WebDSL is stated at WebDSL.org as follows
\begin{quote}
WebDSL is a domain-specific language for developing dynamic web applications with a rich data model.
\footnote{TODO}
\end{quote}

The language supports modeling of
\begin{itemize}
	\item the applications domain in data entities
	\item the applications user interface
	\item the page flow
	\item security and access restrictions
	\item dynamic behaviour, commonly refered to as Ajax. 
	\item styling
\end{itemize}

\index{backend} \index{target}
Last but not least, webDSL applications can be targeted to different front- and backends, including a Tomcat servlet, Google Apps Python backend and a Ajax enabled (dynamic pages) or Ajax-disabled (static pages) front-end. 

\section[]{Scientific work}

lijst met verschenen papers
% WebDSL
% 
% Feedback
% 
% WebDSL is a domain-specific language for developing dynamic web applications with a rich data model.
% Features
% 
%     * Domain modeling
%     * Presentation
%     * Page-flow
%     * Access control
%     * Workflow (under construction)
% 
% Software
% 
%     * WebDSL applications are translated to Java webapplications, building on the JSF, Hibernate, and Seam frameworks.
%     * The WebDSL generator is implemented using Stratego/XT and SDF.
%     * Deployment is realized with the Nix software deployment system.
% 
% Release
% 
%     * First alpha release December 2007.
%     * Download
%     * Give us feedback
% 
% Course
% 
% This year's course on program transformation and generation treats WebDSL as a case study of a program generator.
% 
% More details can be found on the program transformation course page.
% Documentation
% 
%     * Getting started
%     * Installing WebDSL
%     * Language documentation
% 
% Publications
% 
%     *
% 
%       E. Visser. "Domain-Specific Language Engineering." In R. Lämmel and J. Saraiva, editors, Proceedings of the Summer School on Generative and Transformational Techniques in Software Engineering (GTTSE'07), Lecture Notes in Computer Science. Springer Verlag, Braga, Portugal, July 2007. (invited tutorial; under construction)
%     *
% 
%       Z. Hemel, L. C. L. Kats, and E. Visser. "Code Generation by Model Transformation" In International Conference on Model Transformation (ICMT'08).
%     *
% 
%       D. Groenewegen and E. Visser. "Declarative Access Control for WebDSL: Combining Language Integration and Separation of Concerns" In International Conference on Web Engineering (ICWE'08) Yorktown Heights, New York, July 2008.
% 
% Developers
% 
% WebDSL is being developed by Eelco Visser and (Ph.D.) students in the context of the Model-Driven Software Evolution project at Delft University of Technology.
% 
%     * Eelco Visser
%     * Zef Hemel (Zef's Development Blog)
%     * Danny Groenewegen
%     * Jippe Holwerda
%     * Lennart Kats
%     * Sander Vermolen
%     * Sander van der Burg
% 



\chapter[installuser]{Installing WebDSL for WebDSL users}
This section describes how to install WebDSL if it is only to be used to compile WebDSL applications. Currently the compiler is only available for Unix based systems, like Linux and Apple's OS-X. The required packages depend on the target platform of WebDSL applications. The WebDSL compiler itself has the following dependencies:
\begin{itemize}
	\item The \emph{gcc} C compiler
\end{itemize}

\section{A note on Nix deployment system}\label{installnix}
\index{Nix deployment system}
WebDSL and Stratego/XT (one of the depencies when extending WebDSL) can be installed manually or using the \emph{Nix}\footnote{http://www.nixos.org} deployment system, which is highly recommended. Nix takes care of depencies and provides the functionality to easily obtain the latest version from the buildfarms.

The latest release of Nix can be found at its website \emph{http://www.nixos.org}. Use the \emph{configure}, \emph{make} and \emph{make install} commands to install the deployment system. As a final step a autostart instruction needs to be added to your \emph{~.profile} file:
\begin{shell}
. /nix/etc/profile.d/nix.sh
\end{shell}

\section{Installing the WebDSL compiler}\label{retreivewebdsl}
Two ways exist to obtain WebDSL, either from the Nix channel, or downloading the sources directly. The latest source can be downloaded from \emph{http://buildfarm.st.ewi.tudelft.nl/releases/strategoxt/full-index-webdsls.html\#webdsls-Unstable}.

\index{WebDSL deployment channel}
Using Nix, one can subscribe to the WebDSL channel using the following code: 
\begin{shell}
nix-channel --add http://buildfarm.st.ewi.tudelft.nl/releases/strategoxt/channels-v3/webdsls-unstable
nix-channel --update
nix-env -i webdsl-8.3pre1057
\end{shell}
After subscribing to the channel updating to the latest version can be done simply using the following commands:
\begin{shell}
nix-channel --update
nix-env -u webdsls
\end{shell}
After obtaining the sources, executing the well-known commands \emph{configure}, \emph{make} and \emph{make install} should do the job. Test your installation by building a first WebDSL application, as described in \ref{firstapp}. Note that your test application cannot be deployed until the instructions in the next section are executed. 

\subsection{Installing the WebDSL backends}\label{installbackend}
Currently, webdsl applications can be compiled to Java Servlets hosted by a Tomcat server\footnote{http://tomcat.apache.org} or to a Python script which can be hosted by Google's AppEngine\footnote{http://code.google.com/appengine/}. 

\subsubsection{Installing the Tomcat backend}
\index{backend, Tomcat}

\subsubsection{Installing the Python backend}
\index{backend, Python}


\chapter{Installing WebDSL for WebDSL developers}
\index{Stratego/XT}
To build your own version of WebDSL, it takes some extra dependencies that needs to be satifisfied before being able to build the compiler. WebDSL has ben beeld on the program transformation toolset \emph{Stratego/XT}\footnote{http://www.strategoxt.org}. First off all, make sure the depencies of Stratego/XT are satisfied. The following linux packages are needed:
\begin{itemize}
	\item install 
	\item curl 
	\item m4 
	\item autoconf 
	\item automake 
	\item pkgconfig 
	\item libtool 
	\item subversion
\end{itemize}

\index{Stratego/XT channel}
Stratego/XT can be installed using the Nix distributed system. Make sure Nix is installed as described in \ref{installnix}. To install Stratego/XT open a console and execute the following commands:
\begin{shell}
nix-channel --add http://releases.strategoxt.org/strategoxt-packages/channels/strategoxt-packages-stable
nix-channel --add http://releases.strategoxt.org/strategoxt/channels/strategoxt-unstable
nix-channel --update
\end{shell}
Then install the stratego packages:
\begin{shell}
nix-env -i aterm java-front sdf2-bundle stratego-shell strategoxt strategoxt-utils
\end{shell}
Finally, update your \emph{~/.profile} again and add:
\begin{shell}
export PKG_CONFIG_PATH=~/.nix-profile/lib/pkgconfig
\end{shell}

\index{WebDSL subversion repository}
The full sources of WebDSL can be retreived from the svn repository at \emph{ https://svn.strategoxt.org/repos/WebDSL/webdsls}. Use the following command to obtain the sources:
\begin{shell}
svn co  https://svn.strategoxt.org/repos/WebDSL/webdsls/trunk
\end{shell}
Finally, you can compile your own webdsl compiler using 
\begin{shell}
cd webdsls
./bootstrap
./configure --disable-shared --prefix=/usr/local
make
make install
\end{shell}
Install any required backends as described in \ref{instalbackend} and check your installation as described in \ref{firstapp}

\index{Macintosh}
\subsection{A note on Mac Users}
Macintosh users require to install some applications already availalble in most linux environments. First install Apple \emph{XCode} from your DVD. Secondly download and install \emph{DarwinPorts}. Follow the instructions and then install the required ports:
\begin{shell}
port install curl m4 autoconf automake pkgconfig libtool subversion
\end{shell}


\chapter[firstapp]{Running your first application }
% Getting Started
% 
% Installation
% 
% See the installation page.
% Using WebDSL for the first time
% 
% The webdsl generator script can generate a default "hello world" application:
% 
% $ webdsl new
% 
% This can be built using the build command:
% 
% $ webdsl build
% 
% This can then be deployed using the deploy command (replacing any existing deployment):
% 
% $ webdsl deploy
% 
% To clean all generated files:
% 
% $ webdsl clean
% 
% For further documentation of the webdsl script command, see webdsl help.
% 
% To run JBoss, enter the JBoss /bin/ directory and enter the following command:
% 
% $ export JAVA_OPTS="-server -Xms40m -Xmx1024m -XX:MaxPermSize=256m -XX:+CMSPermGenSweepingEnabled
%                     -XX:+CMSClassUnloadingEnabled -Xverify:none -Xss10m"
% $ ./run.sh -b 127.0.0.1 -Dbind.address=127.0.0.1
% 
% This runs JBoss in a Java Virtual Machine configured to handle the memory usage involved with (re)deploying large applications. Still, it is possible to receive out-of-memory errors in JBoss; memory leaks unfortunately seem to be a common problem with the application server.
% Documentation
% 
% See the language documentation for more information on the WebDSL language.
% Support
% 
% Found any bugs? Questions? Try the issue tracker, or see the home page for contact information. The developers usually hang out in the #webdsl irc channel on irc.freenode.net (web version), they will gladly help you out in getting started with webDSL and are able to give you up-to-date info on the status of the whole project (i.e. which revision you should use ;) ).



%%Welcome, when you are reading this, you are at the point of solving a recursive problem!
\chapter{Compiling this documentation}
The sources of this documentation can be found in the ./doc/langref/ directory of the WebDSL sources. Compiling the document requires Latex[TODO] and Rubber[TODO] to be installed. After updating the WebDSL source to the most recent version, you may compile your own documentation postscript with the command
\lstset{language={},tabsize=4}
\begin{lstlisting}
rubber -f -s --inplace -d index.tex
\end{lstlisting}


\part{Module Reference}
\chapter{Introduction to WebDSL}
\section[]{An introduction to Domain Specific Languages}
%%Much taken from            WebDSL: A Case Study inDomain-Specific Language Engineering
\index{Domain Specific Language (DSL)}
Software Engineering can be summarized as the struggle for automating or optimizing the software development process. Main goals of software engineering are the improvement of development time, reducing the efforts in software maintanaince and increasing software quality. Since the early days of computer science abstractions have been build on top of other abstractions, leading away from the details of processor instructions, memory management etc. etc., in an attempt to reduce the number of concerns a software developer has to deal with. Those abstraction have led to the huge amount of GPL languages nowadays exist in industry, like Pascal, C, Python, Java and many many more. 
\\
To improve the improve the capabilities all GPL languages support notions like 'libaries' and 'API's', to enable reuse of specific functionality. However a GPL languages have a set of restrictions, which restrain to easily abstract from the GPL, leaving the developer with lots of boilerplate code that has to be written (like initializing database transactions, write conversion mechanims etc). The restrictions created by the nature of GPL's can be summarized as follow:
\\
\begin{description}
	\item[The inexstensibilty of syntax]
	Libraries offer the capability to automate specific semantic behaviour, like querying a database. However, no syntactic extensions or embeddings are possible, to be able to define SQL queries in an intuitive manner. One cannot avoid the usage of strings or other kinds of code-sugar. 
	\item[The absence of domain restrictions]
	In a GPL one can express in every function or class everything that can be expressed in the GPL as a whole. This makes GPL's very expressive in general, it always enables unstructured, semantic incorrect or very indepent code.
	\item[The huge amount of code needed for general patterns]
	Within a certain domain, certain actions or patterns have to be executed frequently. Little abstractions can be made using helper or library functions, inheritance or even macro's to reuse as much code as possible. However the ways abstractions can be made are limited, and not every abstraction can be expressed in the language
	\item[The huge depencens on the target GPL]
	Once a application has been developed in a certain GPL, it takes much effort to rebuild the application in another GPL, even when the language abstractions, semantics and syntax are very similir (e.g. Java and C\#). This show how much code relies on the language it is written for, even when the abstractions and semantics required by that language are very similar to the ones made available with another GPL. 
\end{description}

To deal with the probles stated above, much effort has been put in Model Driven Engineering. Goal is to have a model describing an application, abstraction from the details of programming languages. Domain Specific Languages provide an interface to a developer where only behaviour related to the domain should be specificied, while the compiler, code generator or model interpreter takes care of the boilerplate code needed for a fully working application, query or whatever the target of a DSL. In [dezelfde paper dus] a DSL is defined as
\begin{quote}
A domain-specific language (DSL) is a high-level software implementa
tion language that supports concepts and abstractions that are related
to a particular (application) domain.\footnote{TODO}
\end{quote}
\section[]{Purpose of WebDSL}
WebDSL is an attempt to create a domain specific language for web-applications. Things like peristency, request handling and GPL/ Markup code generation will be taken care of by the language compiler. The responsiblility of the user is to think about the data model, navigation, page layout and such issues. No boilerplate code should be needed to be written by the user, and the way to define te model (textual in this case) should feel intuitive. The purpose of WebDSL is stated at WebDSL.org as follows
\begin{quote}
WebDSL is a domain-specific language for developing dynamic web applications with a rich data model.
\footnote{TODO}
\end{quote}

The language supports modeling of
\begin{itemize}
	\item the applications domain in data entities
	\item the applications user interface
	\item the page flow
	\item security and access restrictions
	\item dynamic behaviour, commonly refered to as Ajax. 
	\item styling
\end{itemize}

\index{backend} \index{target}
Last but not least, webDSL applications can be targeted to different front- and backends, including a Tomcat servlet, Google Apps Python backend and a Ajax enabled (dynamic pages) or Ajax-disabled (static pages) front-end. 

\section[]{Scientific work}

lijst met verschenen papers
% WebDSL
% 
% Feedback
% 
% WebDSL is a domain-specific language for developing dynamic web applications with a rich data model.
% Features
% 
%     * Domain modeling
%     * Presentation
%     * Page-flow
%     * Access control
%     * Workflow (under construction)
% 
% Software
% 
%     * WebDSL applications are translated to Java webapplications, building on the JSF, Hibernate, and Seam frameworks.
%     * The WebDSL generator is implemented using Stratego/XT and SDF.
%     * Deployment is realized with the Nix software deployment system.
% 
% Release
% 
%     * First alpha release December 2007.
%     * Download
%     * Give us feedback
% 
% Course
% 
% This year's course on program transformation and generation treats WebDSL as a case study of a program generator.
% 
% More details can be found on the program transformation course page.
% Documentation
% 
%     * Getting started
%     * Installing WebDSL
%     * Language documentation
% 
% Publications
% 
%     *
% 
%       E. Visser. "Domain-Specific Language Engineering." In R. Lämmel and J. Saraiva, editors, Proceedings of the Summer School on Generative and Transformational Techniques in Software Engineering (GTTSE'07), Lecture Notes in Computer Science. Springer Verlag, Braga, Portugal, July 2007. (invited tutorial; under construction)
%     *
% 
%       Z. Hemel, L. C. L. Kats, and E. Visser. "Code Generation by Model Transformation" In International Conference on Model Transformation (ICMT'08).
%     *
% 
%       D. Groenewegen and E. Visser. "Declarative Access Control for WebDSL: Combining Language Integration and Separation of Concerns" In International Conference on Web Engineering (ICWE'08) Yorktown Heights, New York, July 2008.
% 
% Developers
% 
% WebDSL is being developed by Eelco Visser and (Ph.D.) students in the context of the Model-Driven Software Evolution project at Delft University of Technology.
% 
%     * Eelco Visser
%     * Zef Hemel (Zef's Development Blog)
%     * Danny Groenewegen
%     * Jippe Holwerda
%     * Lennart Kats
%     * Sander Vermolen
%     * Sander van der Burg
% 



\chapter[installuser]{Installing WebDSL for WebDSL users}
This section describes how to install WebDSL if it is only to be used to compile WebDSL applications. Currently the compiler is only available for Unix based systems, like Linux and Apple's OS-X. The required packages depend on the target platform of WebDSL applications. The WebDSL compiler itself has the following dependencies:
\begin{itemize}
	\item The \emph{gcc} C compiler
\end{itemize}

\section{A note on Nix deployment system}\label{installnix}
\index{Nix deployment system}
WebDSL and Stratego/XT (one of the depencies when extending WebDSL) can be installed manually or using the \emph{Nix}\footnote{http://www.nixos.org} deployment system, which is highly recommended. Nix takes care of depencies and provides the functionality to easily obtain the latest version from the buildfarms.

The latest release of Nix can be found at its website \emph{http://www.nixos.org}. Use the \emph{configure}, \emph{make} and \emph{make install} commands to install the deployment system. As a final step a autostart instruction needs to be added to your \emph{~.profile} file:
\begin{shell}
. /nix/etc/profile.d/nix.sh
\end{shell}

\section{Installing the WebDSL compiler}\label{retreivewebdsl}
Two ways exist to obtain WebDSL, either from the Nix channel, or downloading the sources directly. The latest source can be downloaded from \emph{http://buildfarm.st.ewi.tudelft.nl/releases/strategoxt/full-index-webdsls.html\#webdsls-Unstable}.

\index{WebDSL deployment channel}
Using Nix, one can subscribe to the WebDSL channel using the following code: 
\begin{shell}
nix-channel --add http://buildfarm.st.ewi.tudelft.nl/releases/strategoxt/channels-v3/webdsls-unstable
nix-channel --update
nix-env -i webdsl-8.3pre1057
\end{shell}
After subscribing to the channel updating to the latest version can be done simply using the following commands:
\begin{shell}
nix-channel --update
nix-env -u webdsls
\end{shell}
After obtaining the sources, executing the well-known commands \emph{configure}, \emph{make} and \emph{make install} should do the job. Test your installation by building a first WebDSL application, as described in \ref{firstapp}. Note that your test application cannot be deployed until the instructions in the next section are executed. 

\subsection{Installing the WebDSL backends}\label{installbackend}
Currently, webdsl applications can be compiled to Java Servlets hosted by a Tomcat server\footnote{http://tomcat.apache.org} or to a Python script which can be hosted by Google's AppEngine\footnote{http://code.google.com/appengine/}. 

\subsubsection{Installing the Tomcat backend}
\index{backend, Tomcat}

\subsubsection{Installing the Python backend}
\index{backend, Python}


\chapter{Installing WebDSL for WebDSL developers}
\index{Stratego/XT}
To build your own version of WebDSL, it takes some extra dependencies that needs to be satifisfied before being able to build the compiler. WebDSL has ben beeld on the program transformation toolset \emph{Stratego/XT}\footnote{http://www.strategoxt.org}. First off all, make sure the depencies of Stratego/XT are satisfied. The following linux packages are needed:
\begin{itemize}
	\item install 
	\item curl 
	\item m4 
	\item autoconf 
	\item automake 
	\item pkgconfig 
	\item libtool 
	\item subversion
\end{itemize}

\index{Stratego/XT channel}
Stratego/XT can be installed using the Nix distributed system. Make sure Nix is installed as described in \ref{installnix}. To install Stratego/XT open a console and execute the following commands:
\begin{shell}
nix-channel --add http://releases.strategoxt.org/strategoxt-packages/channels/strategoxt-packages-stable
nix-channel --add http://releases.strategoxt.org/strategoxt/channels/strategoxt-unstable
nix-channel --update
\end{shell}
Then install the stratego packages:
\begin{shell}
nix-env -i aterm java-front sdf2-bundle stratego-shell strategoxt strategoxt-utils
\end{shell}
Finally, update your \emph{~/.profile} again and add:
\begin{shell}
export PKG_CONFIG_PATH=~/.nix-profile/lib/pkgconfig
\end{shell}

\index{WebDSL subversion repository}
The full sources of WebDSL can be retreived from the svn repository at \emph{ https://svn.strategoxt.org/repos/WebDSL/webdsls}. Use the following command to obtain the sources:
\begin{shell}
svn co  https://svn.strategoxt.org/repos/WebDSL/webdsls/trunk
\end{shell}
Finally, you can compile your own webdsl compiler using 
\begin{shell}
cd webdsls
./bootstrap
./configure --disable-shared --prefix=/usr/local
make
make install
\end{shell}
Install any required backends as described in \ref{instalbackend} and check your installation as described in \ref{firstapp}

\index{Macintosh}
\subsection{A note on Mac Users}
Macintosh users require to install some applications already availalble in most linux environments. First install Apple \emph{XCode} from your DVD. Secondly download and install \emph{DarwinPorts}. Follow the instructions and then install the required ports:
\begin{shell}
port install curl m4 autoconf automake pkgconfig libtool subversion
\end{shell}


\chapter[firstapp]{Running your first application }
% Getting Started
% 
% Installation
% 
% See the installation page.
% Using WebDSL for the first time
% 
% The webdsl generator script can generate a default "hello world" application:
% 
% $ webdsl new
% 
% This can be built using the build command:
% 
% $ webdsl build
% 
% This can then be deployed using the deploy command (replacing any existing deployment):
% 
% $ webdsl deploy
% 
% To clean all generated files:
% 
% $ webdsl clean
% 
% For further documentation of the webdsl script command, see webdsl help.
% 
% To run JBoss, enter the JBoss /bin/ directory and enter the following command:
% 
% $ export JAVA_OPTS="-server -Xms40m -Xmx1024m -XX:MaxPermSize=256m -XX:+CMSPermGenSweepingEnabled
%                     -XX:+CMSClassUnloadingEnabled -Xverify:none -Xss10m"
% $ ./run.sh -b 127.0.0.1 -Dbind.address=127.0.0.1
% 
% This runs JBoss in a Java Virtual Machine configured to handle the memory usage involved with (re)deploying large applications. Still, it is possible to receive out-of-memory errors in JBoss; memory leaks unfortunately seem to be a common problem with the application server.
% Documentation
% 
% See the language documentation for more information on the WebDSL language.
% Support
% 
% Found any bugs? Questions? Try the issue tracker, or see the home page for contact information. The developers usually hang out in the #webdsl irc channel on irc.freenode.net (web version), they will gladly help you out in getting started with webDSL and are able to give you up-to-date info on the status of the whole project (i.e. which revision you should use ;) ).



%%Welcome, when you are reading this, you are at the point of solving a recursive problem!
\chapter{Compiling this documentation}
The sources of this documentation can be found in the ./doc/langref/ directory of the WebDSL sources. Compiling the document requires Latex[TODO] and Rubber[TODO] to be installed. After updating the WebDSL source to the most recent version, you may compile your own documentation postscript with the command
\lstset{language={},tabsize=4}
\begin{lstlisting}
rubber -f -s --inplace -d index.tex
\end{lstlisting}



\backmatter
\appendix
\part{Examples}

\part{A note about Stratego/XT}
\chapter{Introduction to WebDSL}
\section[]{An introduction to Domain Specific Languages}
%%Much taken from            WebDSL: A Case Study inDomain-Specific Language Engineering
\index{Domain Specific Language (DSL)}
Software Engineering can be summarized as the struggle for automating or optimizing the software development process. Main goals of software engineering are the improvement of development time, reducing the efforts in software maintanaince and increasing software quality. Since the early days of computer science abstractions have been build on top of other abstractions, leading away from the details of processor instructions, memory management etc. etc., in an attempt to reduce the number of concerns a software developer has to deal with. Those abstraction have led to the huge amount of GPL languages nowadays exist in industry, like Pascal, C, Python, Java and many many more. 
\\
To improve the improve the capabilities all GPL languages support notions like 'libaries' and 'API's', to enable reuse of specific functionality. However a GPL languages have a set of restrictions, which restrain to easily abstract from the GPL, leaving the developer with lots of boilerplate code that has to be written (like initializing database transactions, write conversion mechanims etc). The restrictions created by the nature of GPL's can be summarized as follow:
\\
\begin{description}
	\item[The inexstensibilty of syntax]
	Libraries offer the capability to automate specific semantic behaviour, like querying a database. However, no syntactic extensions or embeddings are possible, to be able to define SQL queries in an intuitive manner. One cannot avoid the usage of strings or other kinds of code-sugar. 
	\item[The absence of domain restrictions]
	In a GPL one can express in every function or class everything that can be expressed in the GPL as a whole. This makes GPL's very expressive in general, it always enables unstructured, semantic incorrect or very indepent code.
	\item[The huge amount of code needed for general patterns]
	Within a certain domain, certain actions or patterns have to be executed frequently. Little abstractions can be made using helper or library functions, inheritance or even macro's to reuse as much code as possible. However the ways abstractions can be made are limited, and not every abstraction can be expressed in the language
	\item[The huge depencens on the target GPL]
	Once a application has been developed in a certain GPL, it takes much effort to rebuild the application in another GPL, even when the language abstractions, semantics and syntax are very similir (e.g. Java and C\#). This show how much code relies on the language it is written for, even when the abstractions and semantics required by that language are very similar to the ones made available with another GPL. 
\end{description}

To deal with the probles stated above, much effort has been put in Model Driven Engineering. Goal is to have a model describing an application, abstraction from the details of programming languages. Domain Specific Languages provide an interface to a developer where only behaviour related to the domain should be specificied, while the compiler, code generator or model interpreter takes care of the boilerplate code needed for a fully working application, query or whatever the target of a DSL. In [dezelfde paper dus] a DSL is defined as
\begin{quote}
A domain-specific language (DSL) is a high-level software implementa
tion language that supports concepts and abstractions that are related
to a particular (application) domain.\footnote{TODO}
\end{quote}
\section[]{Purpose of WebDSL}
WebDSL is an attempt to create a domain specific language for web-applications. Things like peristency, request handling and GPL/ Markup code generation will be taken care of by the language compiler. The responsiblility of the user is to think about the data model, navigation, page layout and such issues. No boilerplate code should be needed to be written by the user, and the way to define te model (textual in this case) should feel intuitive. The purpose of WebDSL is stated at WebDSL.org as follows
\begin{quote}
WebDSL is a domain-specific language for developing dynamic web applications with a rich data model.
\footnote{TODO}
\end{quote}

The language supports modeling of
\begin{itemize}
	\item the applications domain in data entities
	\item the applications user interface
	\item the page flow
	\item security and access restrictions
	\item dynamic behaviour, commonly refered to as Ajax. 
	\item styling
\end{itemize}

\index{backend} \index{target}
Last but not least, webDSL applications can be targeted to different front- and backends, including a Tomcat servlet, Google Apps Python backend and a Ajax enabled (dynamic pages) or Ajax-disabled (static pages) front-end. 

\section[]{Scientific work}

lijst met verschenen papers
% WebDSL
% 
% Feedback
% 
% WebDSL is a domain-specific language for developing dynamic web applications with a rich data model.
% Features
% 
%     * Domain modeling
%     * Presentation
%     * Page-flow
%     * Access control
%     * Workflow (under construction)
% 
% Software
% 
%     * WebDSL applications are translated to Java webapplications, building on the JSF, Hibernate, and Seam frameworks.
%     * The WebDSL generator is implemented using Stratego/XT and SDF.
%     * Deployment is realized with the Nix software deployment system.
% 
% Release
% 
%     * First alpha release December 2007.
%     * Download
%     * Give us feedback
% 
% Course
% 
% This year's course on program transformation and generation treats WebDSL as a case study of a program generator.
% 
% More details can be found on the program transformation course page.
% Documentation
% 
%     * Getting started
%     * Installing WebDSL
%     * Language documentation
% 
% Publications
% 
%     *
% 
%       E. Visser. "Domain-Specific Language Engineering." In R. Lämmel and J. Saraiva, editors, Proceedings of the Summer School on Generative and Transformational Techniques in Software Engineering (GTTSE'07), Lecture Notes in Computer Science. Springer Verlag, Braga, Portugal, July 2007. (invited tutorial; under construction)
%     *
% 
%       Z. Hemel, L. C. L. Kats, and E. Visser. "Code Generation by Model Transformation" In International Conference on Model Transformation (ICMT'08).
%     *
% 
%       D. Groenewegen and E. Visser. "Declarative Access Control for WebDSL: Combining Language Integration and Separation of Concerns" In International Conference on Web Engineering (ICWE'08) Yorktown Heights, New York, July 2008.
% 
% Developers
% 
% WebDSL is being developed by Eelco Visser and (Ph.D.) students in the context of the Model-Driven Software Evolution project at Delft University of Technology.
% 
%     * Eelco Visser
%     * Zef Hemel (Zef's Development Blog)
%     * Danny Groenewegen
%     * Jippe Holwerda
%     * Lennart Kats
%     * Sander Vermolen
%     * Sander van der Burg
% 



\chapter[installuser]{Installing WebDSL for WebDSL users}
This section describes how to install WebDSL if it is only to be used to compile WebDSL applications. Currently the compiler is only available for Unix based systems, like Linux and Apple's OS-X. The required packages depend on the target platform of WebDSL applications. The WebDSL compiler itself has the following dependencies:
\begin{itemize}
	\item The \emph{gcc} C compiler
\end{itemize}

\section{A note on Nix deployment system}\label{installnix}
\index{Nix deployment system}
WebDSL and Stratego/XT (one of the depencies when extending WebDSL) can be installed manually or using the \emph{Nix}\footnote{http://www.nixos.org} deployment system, which is highly recommended. Nix takes care of depencies and provides the functionality to easily obtain the latest version from the buildfarms.

The latest release of Nix can be found at its website \emph{http://www.nixos.org}. Use the \emph{configure}, \emph{make} and \emph{make install} commands to install the deployment system. As a final step a autostart instruction needs to be added to your \emph{~.profile} file:
\begin{shell}
. /nix/etc/profile.d/nix.sh
\end{shell}

\section{Installing the WebDSL compiler}\label{retreivewebdsl}
Two ways exist to obtain WebDSL, either from the Nix channel, or downloading the sources directly. The latest source can be downloaded from \emph{http://buildfarm.st.ewi.tudelft.nl/releases/strategoxt/full-index-webdsls.html\#webdsls-Unstable}.

\index{WebDSL deployment channel}
Using Nix, one can subscribe to the WebDSL channel using the following code: 
\begin{shell}
nix-channel --add http://buildfarm.st.ewi.tudelft.nl/releases/strategoxt/channels-v3/webdsls-unstable
nix-channel --update
nix-env -i webdsl-8.3pre1057
\end{shell}
After subscribing to the channel updating to the latest version can be done simply using the following commands:
\begin{shell}
nix-channel --update
nix-env -u webdsls
\end{shell}
After obtaining the sources, executing the well-known commands \emph{configure}, \emph{make} and \emph{make install} should do the job. Test your installation by building a first WebDSL application, as described in \ref{firstapp}. Note that your test application cannot be deployed until the instructions in the next section are executed. 

\subsection{Installing the WebDSL backends}\label{installbackend}
Currently, webdsl applications can be compiled to Java Servlets hosted by a Tomcat server\footnote{http://tomcat.apache.org} or to a Python script which can be hosted by Google's AppEngine\footnote{http://code.google.com/appengine/}. 

\subsubsection{Installing the Tomcat backend}
\index{backend, Tomcat}

\subsubsection{Installing the Python backend}
\index{backend, Python}


\chapter{Installing WebDSL for WebDSL developers}
\index{Stratego/XT}
To build your own version of WebDSL, it takes some extra dependencies that needs to be satifisfied before being able to build the compiler. WebDSL has ben beeld on the program transformation toolset \emph{Stratego/XT}\footnote{http://www.strategoxt.org}. First off all, make sure the depencies of Stratego/XT are satisfied. The following linux packages are needed:
\begin{itemize}
	\item install 
	\item curl 
	\item m4 
	\item autoconf 
	\item automake 
	\item pkgconfig 
	\item libtool 
	\item subversion
\end{itemize}

\index{Stratego/XT channel}
Stratego/XT can be installed using the Nix distributed system. Make sure Nix is installed as described in \ref{installnix}. To install Stratego/XT open a console and execute the following commands:
\begin{shell}
nix-channel --add http://releases.strategoxt.org/strategoxt-packages/channels/strategoxt-packages-stable
nix-channel --add http://releases.strategoxt.org/strategoxt/channels/strategoxt-unstable
nix-channel --update
\end{shell}
Then install the stratego packages:
\begin{shell}
nix-env -i aterm java-front sdf2-bundle stratego-shell strategoxt strategoxt-utils
\end{shell}
Finally, update your \emph{~/.profile} again and add:
\begin{shell}
export PKG_CONFIG_PATH=~/.nix-profile/lib/pkgconfig
\end{shell}

\index{WebDSL subversion repository}
The full sources of WebDSL can be retreived from the svn repository at \emph{ https://svn.strategoxt.org/repos/WebDSL/webdsls}. Use the following command to obtain the sources:
\begin{shell}
svn co  https://svn.strategoxt.org/repos/WebDSL/webdsls/trunk
\end{shell}
Finally, you can compile your own webdsl compiler using 
\begin{shell}
cd webdsls
./bootstrap
./configure --disable-shared --prefix=/usr/local
make
make install
\end{shell}
Install any required backends as described in \ref{instalbackend} and check your installation as described in \ref{firstapp}

\index{Macintosh}
\subsection{A note on Mac Users}
Macintosh users require to install some applications already availalble in most linux environments. First install Apple \emph{XCode} from your DVD. Secondly download and install \emph{DarwinPorts}. Follow the instructions and then install the required ports:
\begin{shell}
port install curl m4 autoconf automake pkgconfig libtool subversion
\end{shell}


\chapter[firstapp]{Running your first application }
% Getting Started
% 
% Installation
% 
% See the installation page.
% Using WebDSL for the first time
% 
% The webdsl generator script can generate a default "hello world" application:
% 
% $ webdsl new
% 
% This can be built using the build command:
% 
% $ webdsl build
% 
% This can then be deployed using the deploy command (replacing any existing deployment):
% 
% $ webdsl deploy
% 
% To clean all generated files:
% 
% $ webdsl clean
% 
% For further documentation of the webdsl script command, see webdsl help.
% 
% To run JBoss, enter the JBoss /bin/ directory and enter the following command:
% 
% $ export JAVA_OPTS="-server -Xms40m -Xmx1024m -XX:MaxPermSize=256m -XX:+CMSPermGenSweepingEnabled
%                     -XX:+CMSClassUnloadingEnabled -Xverify:none -Xss10m"
% $ ./run.sh -b 127.0.0.1 -Dbind.address=127.0.0.1
% 
% This runs JBoss in a Java Virtual Machine configured to handle the memory usage involved with (re)deploying large applications. Still, it is possible to receive out-of-memory errors in JBoss; memory leaks unfortunately seem to be a common problem with the application server.
% Documentation
% 
% See the language documentation for more information on the WebDSL language.
% Support
% 
% Found any bugs? Questions? Try the issue tracker, or see the home page for contact information. The developers usually hang out in the #webdsl irc channel on irc.freenode.net (web version), they will gladly help you out in getting started with webDSL and are able to give you up-to-date info on the status of the whole project (i.e. which revision you should use ;) ).



%%Welcome, when you are reading this, you are at the point of solving a recursive problem!
\chapter{Compiling this documentation}
The sources of this documentation can be found in the ./doc/langref/ directory of the WebDSL sources. Compiling the document requires Latex[TODO] and Rubber[TODO] to be installed. After updating the WebDSL source to the most recent version, you may compile your own documentation postscript with the command
\lstset{language={},tabsize=4}
\begin{lstlisting}
rubber -f -s --inplace -d index.tex
\end{lstlisting}


\part{Architecture of the WebDSL compiler}

\part{Debugging guide to WebDSL applications and the compiler}

\part{Index}
\lstlistoflistings
\printindex
\end{document}
